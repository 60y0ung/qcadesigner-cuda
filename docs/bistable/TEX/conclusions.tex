\chapter{Conclusions}\label{sec:conclusions}
The main result of this work is that the comparison between CPU and GPU code show the speedup obtained, preserving the correctness of the output. Although many improvements has been applied to the simulation code, there are still more that can be done. In particular there still exist the problem of non coalescent accesses to global memory. This limitation is due to the complexity of the neighboring graph structure and can be overcome by using shared memory to store polarizations. Implementing this feature is indeed very hard because the entire mapping of cells on GPU threads have to be changes in order to fit neighboring cells in the same block.\newline
 During the course of this project we got in touch with people from the University of British Columbia which can help solve the problem of non converging iterations. solving this problem will allow to eliminate the need of coloring the circuit and dividing the simulation of each sample in multiple serial simulation of part of the circuits as explained before. We think that this improvement could lead to more parallelism. Although in the current solution the number of max colors is limited by the maximum number of neighbors which is far smaller than the number of cells in the circuit so it doesn't represent a big limitation to the parallelism of the execution.