\chapter{implementation}\label{sec:implementation}
\section{First approach}
The original source code of QCADesigner was downloaded from Mina website (ref). We attempted to make it compile as it was but we did not manage to solve several compilation errors. So we started to focus on the identification of the core algorithm supporting the tool in order to obtain a working batch simulator executable on CPU. This operation took us some weeks of work. Meanwhile we were able to deeply analyze the code. We made some hypothesis on the location of possible bottlenecks, we identified the data structures used to represent circuits and started to consider possible transformations that had to be done in order to obtain fast accessible and light weight data structures allocable on the GPU global memory.

\section{CPU algorithm and profiling}
Bistable engine is thought as a fast and approximated simulation, sufficient to verify the logic functionality of a design. Every cell is represented as a simple two-state system. The entire simulation is divided into samples, that are units of time (not yet experimentally defined), and for each sample the state of each cell is calculated with respect to the other cells within a preset effective radius. This operation is iterated until all cells have converged within a predetermined tolerance. Once the entire system converges the output is recorded and the computation goes on with the next sample, after having updated the input cells with new input values.
The number of samples required to have a good approximation is known (ref) to be $2000*2^N$, where $N$ is the number of inputs. The maximum number of iterations allowed for the convergence within a sample is $100$. Thus, during a simulation each cell's value is computed sequentially $It*2000*2^N$ times, where $It$ is the mean number of iterations needed to reach convergence. The more are the cells, the longer will take each sample to reach convergence.

Once we finished the batch application we started to profile the execution times simulating some circuits. The table (ref) shows 