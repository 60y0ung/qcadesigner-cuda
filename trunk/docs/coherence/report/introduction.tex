\chapter{Project Introduction}\label{sec:intro}
Computer technology has made incredibly progresses since the first general purpose computer was created, and not until nowadays, the annual performance growth has been steadily growing by 35-60\% per year. It has been only recently that this growth has reduced, mainly because the reach of the limits in the miniaturization of silicon-based transistors (and the associated heat disspiation rates and limits into clock frequency improvements), intrinic limits into ILP exploitation and almost unchanged memory latencies. 

In order to overcome some of these problems, in particular those related to the physical limits of silicon-based technologies, a number of approaches are now under development. DNA, membrane and quantum computing are the major areas of reasearch, focusing not only on theoretical, complexity-related aspects of the subject, but also on the physical machineries exploitable to realize these kinds of computations.

In our work we focus on a technology called \textsl{Quantum Dot Cellular Automata cells}, (or, simply, QCA). At the heart of this technology there are quantum mechanics and nanotechnology. The former describes the behaviour of the charges flowing through these cells (\textsl{ie}, the dynamics of this system), while the latter allows for the fabrication of small enough structures capable of maintaining charges in proper position for the required time. 

Even though it has even been demostrated (solely from a mathematical point of view) the the possibility of using QCA cells for building quantum computers - something quite not at hand yet - they can nonetheless be seen as powerful alternatives to silicon-based devices to represent and transmit data. Depending on the fabrication technique employed, in fact, QCAs are theoretically able to switch at orders of magnitude around the THz, to be packaged at very high density levels and to dissipate very small amounts of power (compared to the current silicon-based technology). This, along with the existance of demo systems already proven working, justifies their current and future development as an attractive way of improving nowadays physical computing bottlenecks.

One of the institutions interested and involved into the development of this technology, among the others, is the University of Columbia, and in particular its Microsystems and Nanotechnology Group (MiNa Group). The foundation of our work is their implementation of a simulator (both logical and physical) of QCA based cicuits, QCADesigner. 

Our focus is set on the improvement of the performance of this application (objective: execution time) to the point that is feasible to simulate sequential, large sized circuits.

