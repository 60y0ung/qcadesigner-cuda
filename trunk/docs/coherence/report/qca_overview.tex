\section{Qca Overview}\label{sec:i}
Quantum Dot Cellular Automata (QCA) are a quantum extension to the classical notion of Cellular Automata. 

In computability theory, we identify a Cellular Automata (CA) as a Turing-complete abstract machine model consisting of a grid of cells each of which can be in any of a finite number of states. For every cell in the grid, moreover, we define its neighborhood, which is the set of "near enough" cells. The evolution of this automaton is defined by the transition of the state of the cells, which is a function of both the status of the cell and the status of its neighborhood.

What makes a QCA different from a classical CA is that ... (tunnel effect, quantum mechanics)

Let's take a closer look to a QCA. As you can see in figure{dot} a QCA is made of four components: potential barriers, electron wells, crystalline substrate and electrons themselves. (positioning vs tunnel effect vs clock phase) ...

Apart from the possibility of using QCA for building quantum computing, which is something quite not at hand yet, QCAs can be seen as a powerful alternatives to silicon-based devices to represent data. Depending on the fabrication technique employed, in fact, QCAs are theorethically able to switch at an order of magnitude around the Thz, to be packaged at very high density levels and to disspate very small amounts of power. This, along with the existance of demo systems already proven working, justifies their current and future development as an attractive way of improving nowadays computing performance bottlenecks.

One of the universities inerested and involved into the development of this technology, among the others, is the University of Columbia, and in particular its Microsystems and Nanotechnology Group (MiNa Group). The foundation of our work is their implementation of a simulator (both logical and physical) of QCA based cicuits and is the subject of the next section.
