\chapter{Introduction}\label{chap:intro}
During the last years the main effort of companies interested in developing information technology has been in the direction of shrinking size of transistors, the main component of any modern computer. This effort allowed to build machines able to run at very high frequency with a huge number of transistors on a single chip.  This trend can't go on forever. In the search for a new computational model we focused our attention on Quantum Dot Cellular Automata (QCA), a novel nanostructure-compatible computation paradigm based on arrays of quantum-dot cells to implement digital logic functions.\newline
The  Microsystems and Nanotechnology Group (MiNa) of the University of Columbia (CA) is one of the most interested group in this area. They recently developed a tool able to design QCA circuit and simulate their evolution. Unfortunately the simulator execution time is negatively affected by the number of cells that have to be simulated.
The objective of this project is to improve performances of this tool, exploiting CUDA (Compute Unified Device Architecture) technology, a novel architecture for parallel computation developed by NVIDIA.