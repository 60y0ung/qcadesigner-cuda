\section{CUDA}
\label{sec:art_of_cuda}
   In recent times, Graphics Processing Units (GPUs) have been considered a potential source of computational
power for non-graphical applications, due to the ongoing evolution of their programming interfaces and their
appealing cost-performance figures of merit. Recent works had first attempted to adapt general purpose� applications
to the graphic rendering APIs (OpenGL and DirectX), which up to two years ago represented the only interface to tap
into the GPU computational resources. Tesla is NVIDIA's first dedicated General Purpose GPU.
\newline
\subsection{NVIDIA Tesla Architecture}
   Modern GPUs include hundreds of processing elements. The NVIDIA Tesla GPU series provide a set
of independent multithreaded streaming multiprocessors. \figurename~\ref{fig:LT} shows an overview of the NVIDIA
Tesla streaming processors array which is the part of the GPU architecture responsible for the general purpose computation.
Each streaming multiprocessor is composed by a set of eight streaming processors,two special functional units and a multithreaded
instruction issue unit (respectively indicated as SP, SFU and MT-Issue in \figurename~\ref{fig:Tesla}).\newline
A SP is a fully pipelined single-issue processing core with two ALUs and a single floating point unit (FPU). 
SFUs are dedicated to the computation of transcendental functions and pixel/vertex manipulations.
The MT-Issue unit is in charge of mapping active threads on the available SPs.\newline
  A multiprocessor is able to concurrently execute groups of 32 threads called warps. Since each thread
in a warp has its own control flow, their execution paths may diverge due to the independent evaluation
of conditional statements. In this case, the warp serially executes each path. When the warp is executing
a given path, all threads that have not taken that path are disabled. On the other hand, in case the
control flows converge again, the warp may return to a single, parallel execution of all threads.\newline
 Each multiprocessor executes warps in a fashion much like the Single Instruction Multiple Data (SIMD)
paradigm, since every thread will be assigned to a different SP and every active thread will execute the
same instruction on different data.\newline
   The MT-Issue unit weaves threads into a number of warps and schedules an active warp for execution,
using a round-robin scheduling policy with aging for this purpose.\newline
   Streaming multiprocessors are in turn grouped in Texture Processor Clusters (TPC). Each TPC includes three streaming
multiprocessors in the Tesla architecture. The TPC also includes support for Texture processing, though these features are
seldom used for general purpose computing and will not be investigated in this description.\newline
   Finally, the NVIDIA GPU on-board memory hierarchy includes registers (private to each SP), on-chip memory and off-chip memory.
The on-chip memory is private to each multiprocessor, and is split into a very small instruction cache, a read-only data cache,
and 16 KB of addressable shared data, respectively indicated as I-cache, C-cache and Shared Memory in \figurename~\ref{fig:Tesla}.
This shared memory is organized in 16 banks that can be concurrently accessed, each bank having a single read/write port.\newline
  The GPU we used for our project is the NVIDIA Tesla c1060. 
 With the computational power of its 240 Streaming Processors (grouped into 30 TPCs) and the 102,4 GB/s max bandwidth of its 4096 MB
GDDR3 memory, it represents one of the most performing GPUs on the market.\newline

\subsection{CUDA programming model}
   The Compute Unified Device Architecture (CUDA), proposed by NVIDIA for its GeForce (8 series and above), Quadro and Tesla
graphics processors, exposes a programming model that integrates host and GPU code in the same C/C++ source files.\newline
The main programming structure supporting parallelism is an explicitly parallel function invocation (kernel) which
is supposed to be executed by a user-specified number of threads.
Every kernel is explicitly invoked by host code and executed by the device, while the host-side code continues execution
 asynchronously after instantiating the kernel. The programmer is provided with a specific synchronizing function call
to wait for the completion of the active asynchronous kernel computation.\newline
   The CUDA programming model abstracts the actual parallelism implemented by the hardware architecture,
providing the concepts of block and thread to express concurrency in algorithms.
A block captures the notion of a group of concurrent threads. Blocks are required to execute independently, so that it has
to be possible to execute them in any order (in parallel or in sequence). Therefore, the synchronization primitives semantically
act only among threads belonging to the same block.
Intra-block communications among threads use the logical shared memory associated with that block.\newline
   Since the architecture does not provide support for the message-passing techniques, threads belonging
to different blocks must communicate through global memory. The global memory is entirely mapped to the off-chip memory.
The concurrent accesses to logical shared memory by threads executing within the same block are supported through an explicit
barrier synchronization primitive.\newline
   A kernel call-site must specify the number of blocks as well as the number of threads within each
block when executing the kernel code. The current CUDA programming model imposes a capping of 512 threads per block.\newline
   The mapping of threads to processors and of blocks to multiprocessors is mainly handled by hardware controller components.
Two or more blocks may share the same multiprocessor through mechanisms that allow fast context switching depending on 
the computational resources used by threads and on the constraints of the hardware architecture.
The number of concurrent blocks managed by a single multiprocessor is currently limited to 8.\newline
   In addition to the logical shared memory and the global memory, in the CUDA programming model each thread may access a constant memory.
 An access to this read-only memory space is faster than one to global memory, provided that there is sufficient access locality since
constant memory is implemented as a region of global memory fit with an on-chip cache.
Finally, another portion of the off-chip memory may be allocated as a local memory that is used as thread private resource.
Since the local memory access is slow, the shared memory also serves as an explicitly managed cache though it is up to the
programmer to warrant that the local data being saved in shared memory are not accessed by other
threads. Shared memory comes in limited amounts (threads within each block typically share 16 KB of memory) hence,
it is crucial for performance that each thread handle only small chunks of data.

\subsection{SIMT and SIMD}
   SIMT architecture is similar to single instruction, multiple-data (SIMD) design, which applies one instruction to multiple data lanes.\newline
 The difference is that SIMT applies one instruction to multiple independent threads in parallel, not just multiple data lanes.
A SIMD instruction controls a vector of multiple data lanes together and exposes the vector width to the software, whereas a SIMT instruction
controls the execution and branching behavior of one thread.\newline
   In contrast to SIMD vector architectures, SIMT enables programmers to write thread level parallel code for independent threads
as well as data-parallel code for coordinated threads.\newline
For program correctness, programmers can essentially ignore SIMT execution attributes such as warps; however, they can
achieve substantial performance improvements by writing code that seldom requires threads in a warp to diverge. In practice, this
is analogous to the role of cache lines in traditional codes: programmers can safely ignore cache line size when designing for
correctness but must consider it in the code structure when designing for peak performance.\newline 
SIMD vector architectures, on the other hand, require the software to manually coalesce loads into vectors and to manually manage
divergence.







