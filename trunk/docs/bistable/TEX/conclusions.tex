\chapter{Conclusions}\label{sec:conclusions}
Our implementation effectively exploits CUDA technology to speedup the execution of the bistable engine simulation preserving the correctness of the results.
Although many improvements has been applied to the simulation code, there are still more that can be done. In particular there still exist the problem of non coalescent accesses to global memory discussed in \ref{sec:optimizations}. This limitation is due to the complexity of the neighboring graph structure and could be overcome by using shared memory, which is also faster than global memory, to store polarizations. Implementing this feature is indeed very hard because the entire mapping of cells on GPU threads and the simulation algorithm have to be changed in order to fit neighboring cells in the same block.\newline
During the course of this project we got in touch with Professor Konrad and a PhD student from the University of British Columbia which is willing to help solve the problem of non converging iterations discussed in \ref{sec:new_algorithm}. solving this problem will allow to eliminate the need of coloring the circuit and dividing the simulation of each sample in multiple serial simulation of part of the circuits. We think that this improvement could lead to more parallelism. Although in the current solution the number of max colors is limited by the maximum number of neighbors which is far smaller than the number of cells in the circuit so it does not represent a severe limitation to the parallelism of the execution.